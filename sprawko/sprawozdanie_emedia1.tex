\documentclass{article}

\usepackage{polski}
\usepackage[utf8]{inputenc}
\usepackage{amsmath}
\usepackage{graphicx}
\usepackage{url}


\title{Praca inżynierska}
\date{2017-10-01}
\author{Jędrzej Kozal, Karol Szpila}

\begin{document}

\begin{titlepage}
	\centering
	\includegraphics[width=0.25\textwidth]{logo_pol_wroclaw.png}\par\vspace{1cm}
	{\scshape\LARGE Politechnika Wrocławska \par}
	\vspace{1cm}
	{\scshape\Large E-media\par}
	\vspace{1.5cm}
	{\huge\bfseries Sprawozdanie z projektu, część 2 \par}
	\vspace{2cm}
	{\Large\itshape Karol Szpila, Jędrzej Kozal\par}
	\vfill
	prowadzący\par
	Dr hab.~Wojciech \textsc{Bożejko}, prof. nadzw. PWr

	\vfill

% Bottom of the page
	{\large 2017-10-01\par}
\end{titlepage}

\section{Wstęp}
Druga część projektu polegała na zaimplementowaniu wybranego algorytmu szyfrowania niesymetrycznego i wykorzystania tego algorytmu zaszyfrowania masy bitowej z poprzedniej części projektu. Weryfikacja poprawności wykonania ćwiczenia miała odbyć się przez wyświetlenie transformaty Fouriera, następnie zaszyfrowanie i odszyfrowanie pliku oraz ponowne wyświetlenie transformaty Fouriera. W ramach tego projektu zaimplementowano algorytm RSA.

\subsection{Wykorzystane narzędzia}
W drugiej częsci projektu zostały wykorzystane wszystkie narzędzia z poprzedniej części. Dodatkowo przy implementacji algorytmu RSA wykorzystano pakiety Crypto.Util oraz fractions do odpowiednio losowania liczb pierwszych oraz wyznaczania największego wspólnego dzielnika.

\subsection{Wstęp teoretyczny}
Algorytm RSA jest algorytmem szyforwania kluczem niesymetrycznym, co oznacza że istnieje para kluczy: publiczny i prywatny. Klucz publiczny jest udostępniany każdemu kto chce wysłać wiadomość do osoby generującej klucze. Zaszyfrowana wiadomość może być odszyfrowywana tylko z wykorzystaniem klucza prywatnego, który nie jest ogólnie dostępny.

Napisać coś po kolei o tych wszystkich parametrach:
p, q - duże liczby pierwsze

Zaszyfrowanie wiadomości T odbywa się w następujący sposób:
\begin{equation}
	E = T^e \; mod \; n
\end{equation}

Następnie odszyfrowanie wiadomości odbywa się zgodnie ze wzorem:
\begin{equation}
	T = E^d \; mod \; n
\end{equation}

Nasuwa się pewien problem zwizany z koniecznością operowania na bardzo dużych liczbach całkowitych. Problem ten można ominąć przez wykorzystanie następujących własności matematycznych:

Przyjmijmy dowolną liczbę całkowitą $a$ postaci:
\begin{equation}
	a = pn + r
\end{equation}
, gdzie r to reszta z dzielenia a przez n. Z definicji mamy więc:
\begin{equation}
	a \; mod \; n = r
\end{equation}

Korzystając z powyższych definicji można z łatwością wyprowadzić:
\begin{equation}
	a^2 = (pn + r)^2 = p^2 n^2 + 2 p n r + r^2
\end{equation}

\begin{equation}
	a^2 \; mod \; n = ((p^2 n + 2 p r) n + r^2) \; mod \; n \\
	= r^2 \; mod \; n
\end{equation}

Jeśli wykładnik we wzorze (2) i (3) zapiszemy w formie binarnej, to powyższy wzór może zostać wykorzystany do generacji kolejnych reszt z dzielenia przez kolejne bity. Takie indukcyjne podejście do problemu podowduje że w każdym kroku mamy do czynienia z liczbami nie większymi niż $n - 1$. Zapsując wykładnik ze wzoru (2) i (3) jako sumę bitów możemy przejść do obliczania reszty z dzielenia iloczynu liczb:

\begin{equation}\label{eq:pareto mle2} 
	\begin{aligned}
		a = b n + r_a \\
		c = d n + r_c	
	\end{aligned}
\end{equation}

\begin{equation*}\label{eq:pareto mle2}
	\begin{aligned}
		a c \; mod \; n = ((b n + r_a)(d n + r_c)) \; mod \; n  \\
		= ((b d n + b r_c + d r_a) n + r_a r_c) \; mod \; n = (r_a r_c) \; mod \; n
	\end{aligned}
\end{equation*}	

Korzystając z powyższych własności i binarnej reprezentacji wykładnika można utworzyć tablicę przechowującą wartości wyrażenia $a^b \; mod \; n$, gdzie b to kolejne potęgi dwójki (kolejne wartości pojedyńczych bitów). Następnie korzystając z równania (7) można iterując po bitach b, wymnarzając kolejne wartości z tablicy (jeśli bit jest równy 1), oraz wykonując działanie modulo otrzymujemy poprawny wynik działania. 


\newpage
\begin{thebibliography}{9}

\bibitem{wav} 
Przykład działania algorytmu RSA,
\\\texttt{http://www.cryer.co.uk/glossary/r/rsa/an\_example\_of\_rsa\_algorithm.html}

\bibitem{bmpB} 
Obliczanie prywatnego wykładnika,
\\\texttt{https://crypto.stackexchange.com/questions/5889/calculating-rsa-\\private-exponent-when-given-public-exponent-and-the-modulus-fact}

\bibitem{bmp} 
Opis rozszerzonego alogrytmu Euklidesa,
\\\texttt{https://en.wikipedia.org/wiki/Extended\_Euclidean\_algorithm}

\bibitem{kivy}
Twierdzenie Eulera,
\\\texttt{https://en.wikipedia.org/wiki/Euler\%27s\_theorem}

\end{thebibliography}

\end{document}